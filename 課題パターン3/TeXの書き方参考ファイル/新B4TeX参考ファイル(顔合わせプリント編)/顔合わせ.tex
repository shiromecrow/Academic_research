%==========================オプションおよび文書クラスの設定==========================
%"autodetect-engine"-どのエンジンでもコンパイル可能にするオプション
%"dvipdfmx-if-dvi"-必要な場合のみdvipdfmx経由のpdf化をするオプション(LuaTeXやXeTeXはPDFに直接変換するため)
%"ja=standard"-日本語文書の標準設定を利用するオプション
%"bxjs…"-どのエンジンでも利用可能なドキュメントクラス
%--以下のいずれかを選択--
\documentclass[autodetect-engine,dvipdfmx-if-dvi,ja=standard,a4paper,11pt]{bxjsarticle} %章の無いレポート
%\documentclass[autodetect-engine,dvipdfmx-if-dvi,ja=standard,a4paper,10pt]{bxjsslide} %スライド
%\documentclass[autodetect-engine,dvipdfmx-if-dvi,ja=standard,a4paper,10pt]{bxjsbook} %書籍
%\documentclass[autodetect-engine,dvipdfmx-if-dvi,ja=standard,a4paper,10pt]{bxjsreport} %章のある論文やレポート

%==============================プリアンブルの設定==============================
\title{研究会資料} %タイトル
\author{B4 浅見憲輝} %著者名
\date{2019.12.19}%日付 %日付下の余白をN[mm]減らす

%///////////////////////////////////////////////////////////////////////////////////////////////////////////
%////////////////////////////////////パッケージの読込み及び設定の書換え//////////////////////////////////////
%///////////////////////////////////////////////////////////////////////////////////////////////////////////
\usepackage{graphicx} %図の挿入に関するパッケージ
\usepackage{float} %[H]で図の位置を固定する機能をONにするパッケージ
\usepackage{subcaption} %サブキャプションに関するパッケージ
\captionsetup{labelsep=space} %サブキャプション後の":"を非表示にする
\usepackage{enumerate} %{enumerate}[]の,[]の中の通りの箇条書きにすることができるパッケージ
\usepackage{amsmath} %数式に関するパッケージ
\usepackage{mathtools} %数式に関するパッケージ
\usepackage{bm} %ベクトル表示のコマンドを追加するパッケージ
\usepackage{comment} %複数行のコメントアウトを可能にするパッケージ
\usepackage{ascmac} %枠に関するパッケージ
\usepackage{tabularx} %表に関するパッケージ
\setpagelayout{top=10truemm,bottom=15truemm,left=15truemm,right=15truemm} 
\usepackage{url}

%余白に関する設定の書換え(bxjs…クラスではgeometryパッケージは使用不可)
\belowcaptionskip=-0pt %キャプション下の余白をN[pt]減らす
\graphicspath{{../figures/}} %図を挿入する際に.texファイルの上の階層にあるfiguresというフォルダを参照可能にする

%使用記号の追加
\newcommand{\divergence}{\mathrm{div}\,}  %ダイバー
\newcommand{\grad}{\mathrm{grad}\,}  %グラディエント
\newcommand{\rot}{\mathrm{rot}\,}  %ローテーション

\pagestyle{myheadings}
\markright{\footnotesize 2月28日(金)15:00~ 顔合わせ}
%================================以下本文================================
\begin{document}
%
%
\section*{◆後藤田研究室へようこそ!!◆}
%
新B4の皆さん,配属おめでとうございます.本研究室を志望していただきありがとうございます!!本研究室で今日から卒業までどうぞ宜しくお願い致します.

\section{本日のながれ}
\begin{enumerate}
  \item 自己紹介(新B4→M1→M2)
  \item 連絡先交換(LINE,学籍番号)
  \item 研究室紹介(研究内容,年間スケジュール,春季課題など)
  \item MATLABのインストール
  \item \LaTeX の紹介・インストール
  \item 自由時間
  \item 懇親会@金盛楼
\end{enumerate}
%
\section{研究室紹介}
後藤田研究室は, 流動, 熱・物質拡散, 化学反応が相互に作用し合う反応系熱流体を対象に, その複雑な非線形ダイナミックスの解明・モデル化, 工学的な応用への実装(例えば, 検知・能動制御システムの開発など)を目指した先導的な基盤研究を複雑系数理の視点から取り組んでいます. これらの研究を通じて, 環境・エネルギー分野の新領域における機械システムの創成に貢献していくことを目指しています\cite{GotoLab}.

B3までに学習した流体力学,熱力学の知識はもちろんのこと,情報理論,統計力学なども必要となるため,本研究室ではB4生に対しはまずはこれらの分野に慣れるため,春季休暇課題や基礎勉強会等の機会を設けています.覚えることはたくさんありますが,どれも世界的に類を見ない最先端の研究分野です!!一緒に頑張っていきましょう.

\section{MATLABのインストール}
本研究室では数値解析やグラフ作成に"MATLAB"と呼ばれるソフトを用います.研究室・院生室内のPCには基本的にインストールされていますが,私用のPCにも入れておくと自宅でもちょっとした作業ができるため便利です.理科大生は学籍番号のアドレスがあれば無料でインストールができます.プログラム言語としてはかなり直感的に操作できるので,春季課題や研究室内で使用されているプログラムの解読,簡単なプログラムの作成などを通じて十分使い方は理解できると思います.
\newpage

\subsection*{◆インストール手順◆}
\begin{itemize}
\item "理科大~matlab"で検索
\item "学生個人所有のPCへのMATLABのダウンロードの公開について(お知らせ) "のページにて,"参照URL"を確認
\item \url{http://www.ed.tus.ac.jp/tusonly/matlab/}にアクセス(学内Wi-Fiにてアクセス可)
\item "アクティベーションキーの取得ページ"にアクセスし,アクティベーションキーを取得する
\item "MathWorksアカウントの取得とインストール(学生個人所有PC用)"の手順に従いインストールする
\end{itemize}

\section{\LaTeX の紹介・インストール}
本研究室では研究会資料や卒業論文などの文章執筆にフリーの文章処理システムである"\LaTeX"を使用しています.この文章も\LaTeX を使用しており,Word等と比べると執筆方法に癖はありますが,数式や参考文献などの文書スタイルが非常に綺麗であることや,学術論文や学会要旨などへの入稿形式の統一が容易であることから,大学のみならず学術分野では広く用いられています.新B4の皆さんには\LaTeX をインストールして頂き,まずは簡単な文章作成に慣れて頂くため,春季課題を\LaTeX で編集し,提出して頂くことを考えています.

\subsection*{◆インストール手順◆}
\begin{itemize}
\item "latex 導入"等で検索し,"簡単LaTeXインストールWindows編(2016年4月版)"\cite{Latex}にアクセスする
\item サイト内の手順に従い,TeXworksを導入する
\item ネットで適当なテンプレートを見つけてきて,春季課題のレポート内容を編集しながら慣れる
\end{itemize}
かなり手抜きな内容になってしまい申し訳ないですが,まずは春季課題レポートの作成から頑張ってみてください.また,研究室・院生室内のPCにもTeXworksおよびTeXstudioがインストールされているものがあります.M1,M2に声をかけて頂ければ対応するので気軽に相談してください.
\newpage

\section{書籍}
今年の基礎研究会では以下の書籍を使用して輪講を行います.購入を強くお勧めします.
\begin{itemize}
\item  東京大学工学教程基礎系数学確率・統計Ⅰ,縄田和満,丸善出版 (2013)\\
※基礎勉強会では"確率の基礎"の章から輪読を開始する予定です.
\end{itemize}
また,本研究室で比較的利用頻度の高い書籍も紹介しておきます.研究室内にも所蔵があるので,使っていて頻度が高ければ購入すると良いと思います.
\begin{itemize}
\item カオス時系列解析の基礎と応用,合原一幸編,産業図書 (2000)\\
絶版.研究室内に数冊所蔵有.
\item 複雑系の数理,松葉育雄,朝倉書店 (2004)\\
3年前の基礎勉強会資料です.研究室内に数冊所蔵有
\item 東京大学工学教程システム工学システム理論I,大橋弘忠,丸善出版 (2015)\\
一昨年の基礎勉強会資料です,研究室内に数冊所蔵有.
\item  東京大学工学教程システム工学システム理論II,大橋弘忠,丸善出版 (2016)\\
去年の基礎勉強会資料です,研究室内に数冊所蔵有.
\item 反応系の流体力学,植田利久,コロナ社 (2002)\\
本研究室で扱う支配方程式がかなりわかりやすく説明されています.1,2冊研究室内にあります.
\item 最新MATLABハンドブック,小林一行,秀和システム (2017)\\
MATLABの初心者にもわかりやすい内容です.研究室内に数冊所蔵有.
\item LaTeX2ε美文書作成入門,奥村晴彦,技術評論社 (2017)\\
\LaTeX の初心者にもわかりやすい内容です.研究室内に数冊所蔵有.
\end{itemize}

\section{連絡先}
わからないことがあれば下記,または知り合いの先輩までご連絡ください.LINEでもOKです.
\begin{itemize}
\item 浅見 \url{4516004@ed.tus.ac.jp}
\end{itemize}

\begin{thebibliography}{9999}
\bibitem{GotoLab}
後藤田研究室HP,\url{https://www.rs.tus.ac.jp/gotodalab/index.html}
\bibitem{Latex}
簡単LaTeXインストールWindows編(2016年4月版),\url{https://did2memo.net/2016/04/24/easy-latex-install-windows-10-2016-04/}
\end{thebibliography}
\newpage

\section*{◆春季課題◆}
後日,必要なファイルをまとめた学習キット(zipファイル)をメールで送信します.A4用紙4枚程度のレポートを\LaTeX で作成するのが春季課題です.人によって内容が異なりますが,研究室の書籍を使ったり,同期内で情報共通したりして取り組んでください.
主に以下の文献が参考になると思います.
\begin{itemize}
\item 後藤田研究室,"順列エントロピーとは"(学習キットに同梱)
\item "ロジスティック方程式とロジスティック写像"(過去の基礎勉強会資料です.学習キットに同梱)
\item  上記の書籍
\end{itemize}

\end{document}